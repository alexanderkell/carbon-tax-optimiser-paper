%%
%% This is file `sample-sigconf.tex',
%% generated with the docstrip utility.
%%
%% The original source files were:
%%
%% samples.dtx  (with options: `sigconf')
%% 
%% IMPORTANT NOTICE:
%% 
%% For the copyright see the source file.
%% 
%% Any modified versions of this file must be renamed
%% with new filenames distinct from sample-sigconf.tex.
%% 
%% For distribution of the original source see the terms
%% for copying and modification in the file samples.dtx.
%% 
%% This generated file may be distributed as long as the
%% original source files, as listed above, are part of the
%% same distribution. (The sources need not necessarily be
%% in the same archive or directory.)
%%
%% The first command in your LaTeX source must be the \documentclass command.
\documentclass[sigconf]{acmart}

%%
%% \BibTeX command to typeset BibTeX logo in the docs
\AtBeginDocument{%
  \providecommand\BibTeX{{%
    \normalfont B\kern-0.5em{\scshape i\kern-0.25em b}\kern-0.8em\TeX}}}

%% Rights management information.  This information is sent to you
%% when you complete the rights form.  These commands have SAMPLE
%% values in them; it is your responsibility as an author to replace
%% the commands and values with those provided to you when you
%% complete the rights form.
\setcopyright{acmcopyright}
\copyrightyear{2020}
\acmYear{2020}
\acmDOI{10.1145/3185768.3186313}

%% These commands are for a PROCEEDINGS abstract or paper.
\acmConference[ICPE '20]{ICPE '20: ACM/SPEC International Conference on Performance Engineering}{April 20--24, 2020}{Edmonton, Canada}
\acmBooktitle{ICPE '20: ACM/SPEC International Conference on Performance Engineering,
  April 20--24, 2020, Edmonton, Canada}
\acmPrice{15.00}
\acmISBN{978-1-4503-XXXX-X/18/06}


%%
%% Submission ID.
%% Use this when submitting an article to a sponsored event. You'll
%% receive a unique submission ID from the organizers
%% of the event, and this ID should be used as the parameter to this command.
%%\acmSubmissionID{123-A56-BU3}

%%
%% The majority of ACM publications use numbered citations and
%% references.  The command \citestyle{authoryear} switches to the
%% "author year" style.
%%
%% If you are preparing content for an event
%% sponsored by ACM SIGGRAPH, you must use the "author year" style of
%% citations and references.
%% Uncommenting
%% the next command will enable that style.
%%\citestyle{acmauthoryear}

%%
%% end of the preamble, start of the body of the document source.
\begin{document}
\emergencystretch 3em

%%
%% The "title" command has an optional parameter,
%% allowing the author to define a "short title" to be used in page headers.
\title{Optimizing carbon tax for decentralized electricity markets with an agent-based model}

%%
%% The "author" command and its associated commands are used to define
%% the authors and their affiliations.
%% Of note is the shared affiliation of the first two authors, and the
%% "authornote" and "authornotemark" commands
%% used to denote shared contribution to the research.
%\author{Alexander Kell}
%\affiliation{%
%  \department{School of Computing}
%  \institution{Newcastle University}
%  \city{Newcastle upon Tyne}
%  \country{UK}
%}
%\email{a.kell2@newcastle.ac.uk}
%
%\author{A. Stephen McGough}
%\affiliation{%
%  \department{School of Computing}
%  \institution{Newcastle University}
%  \city{Newcastle upon Tyne}
%  \country{UK}
%}
%\email{stephen.mcgough@newcastle.ac.uk}
%
%\author{Matthew Forshaw}
%\affiliation{%
%  \department{School of Computing}
%  \institution{Newcastle University}
%  \city{Newcastle upon Tyne}
%  \country{UK}
%}
%\email{matthew.forshaw@newcastle.ac.uk}
\author{Anonymized}
%%
%% By default, the full list of authors will be used in the page
%% headers. Often, this list is too long, and will overlap
%% other information printed in the page headers. This command allows
%% the author to define a more concise list
%% of authors' names for this purpose.
\renewcommand{\shortauthors}{Kell et al.}

%%
%% The abstract is a short summary of the work to be presented in the
%% article.
\begin{abstract}
 
 Placeholder
 
\end{abstract}

%%
%% The code below is generated by the tool at http://dl.acm.org/ccs.cfm.
%% Please copy and paste the code instead of the example below.
%%


%%
%% Keywords. The author(s) should pick words that accurately describe
%% the work being presented. Separate the keywords with commas.
\keywords{Energy markets, policy, carbon tax, genetic algorithm, optimization, digital twin, agent-based models, electricity market model}

%% A "teaser" image appears between the author and affiliation
%% information and the body of the document, and typically spans the
%% page.


%%
%% This command processes the author and affiliation and title
%% information and builds the first part of the formatted document.
\maketitle

\section{Introduction}


% Computer simulation - why it is required

Computer simulation allows practitioners to model real-world systems using software. These simulations allow for `\textit{what-if}' analyses which can provide an indication as to how a system may behave under certain policies, environments and assumptions. These simulations become important in systems which have high costs, impacts or risks associated with them.

% Electricity markets

Electricity markets are an example of such a system. Disruptions to electricity supply, a substantial increase in the cost of electricity or unrestrained carbon emissions have the potential to destabilise economies. It is for reasons such as these that electricity market models are used to test hypotheses, develop strategies and gain an understanding of underlying dynamics to prevent undesirable consequences \cite{Jebaraj2006}. 

In this paper we use the electricity market agent-based model ElecSim to find an optimum carbon tax strategy \cite{Kell}. Specifically, we use a genetic algorithm to find a carbon tax policy to reduce both average electricity price and the relative carbon density by 2035 for the UK electricity market. 

For this, we use the reference scenario projected by the UK Government's Department for Business \& Industrial Strategy (BEIS) and used model parameters calibrated by Kell \textit{et al.} \cite{DBEIS2019,Kell2020}. This reference scenario projects energy and emissions until 2035. 

In contrast to grid or random search, genetic algorithms have the ability to converge on an optimal solution by trialling a fewer number of parameters. This is of particular importance in cases with a large number of parameters or long compute time of the function to optimise.




% Digital twin

% Identify optimal parameters

% Multi-objective and global optima

% Use of NSGA II and GA

% Example of ElecSim

% What we do

% Paper layout


\section{Literature Review}

Placeholder
% Optimisation literature

% Carbon optimisation related literature


\section{Optimization methods}

Placeholder
% GA \cite{John Holland}

% NSGA II


\section{Simulation Environment}

Placeholder
% ElecSim

% Parameters for Carbon Tax


\section{results}
Placeholder

\begin{figure}
\centering
\includegraphics[width=0.49\textwidth]{/Users/b1017579/Documents/PhD/Papers/6-carbon-optimiser/acmart-master/samples/figures/results/free_points/free_points_ga_development.pdf}
\caption{Development of genetic algorithm rewards of average electricity price and relative carbon density in 2035 over time for highest degrees of freedom per year.}
\label{fig:forward_scenario_best_pdcs1}
\end{figure}



\begin{figure}
\centering
\includegraphics[width=0.49\textwidth]{/Users/b1017579/Documents/PhD/Papers/6-carbon-optimiser/acmart-master/samples/figures/results/free_points/best_heatmap.pdf}
\caption{2D density plot of carbon tax strategies that led to an average electricity price of below \textsterling5/MWh by 2035.}
\label{fig:forward_scenario_best_pdcs}
\end{figure}


\begin{figure}
\centering
\includegraphics[width=0.49\textwidth]{/Users/b1017579/Documents/PhD/Papers/6-carbon-optimiser/acmart-master/samples/figures/results/linear/linear_ga_development.pdf}
\caption{Development of genetic algorithm rewards of average electricity price and relative carbon density in 2035 over time for linear carbon strategy.}
\label{fig:forward_scenario_best_pdcs}
\end{figure}


\begin{figure}
\centering
\includegraphics[width=0.49\textwidth,]{/Users/b1017579/Documents/PhD/Papers/6-carbon-optimiser/acmart-master/samples/figures/results/linear/linear_ga_development_distribution.pdf}
\caption{Density plot of average electricity price smaller than \textsterling 8/MWh in 2035 over generation number of genetic algorithm.}
\label{fig:forward_scenario_best_pdcs}
\end{figure}



\begin{figure}
\centering
\includegraphics[width=0.49\textwidth,]{/Users/b1017579/Documents/PhD/Papers/6-carbon-optimiser/acmart-master/samples/figures/results/linear/linear_actual_lines.pdf}
\caption{Linear carbon tax strategies visualised with average electricity price smaller than \textsterling 5/MWh.}
\label{fig:forward_scenario_best_pdcs}
\end{figure}



\section{Conclusion}

Placeholder

%\section{Acknowledgments}



\begin{acks}
Anonymized
%This work was supported by the Engineering and Physical Sciences Research Council, Centre for Doctoral Training in Cloud Computing for Big Data [grant number EP/L015358/1].
\end{acks}

%%
%% The next two lines define the bibliography style to be used, and
%% the bibliography file.
\bibliographystyle{ACM-Reference-Format}
\bibliography{library,custombibtex}

%%
%% If your work has an appendix, this is the place to put it.
\appendix


\end{document}
\endinput
%%
%% End of file `sample-sigconf.tex'.
